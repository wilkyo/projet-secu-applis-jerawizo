\subsection{Signatures}

À la base, nous devions modifier l'application \textbf{Castle Defense}.
Le problème qui s'est rapidement posé est que l'application ne se lançait pas après signature avec une de nos clés.
Ceci s'est expliqué par le fait que l'application possédait une constante permettant de vérifier qu'elle était bien signée par la clé du développeur,
empêchant ainsi toute recompilation par une personne tierce.

Lors d'une séance de travaux pratiques, nous avons pris connaissance d'un moyen de passer outre cette vérification en utilisant une preuve de réalisation
(proof of concept \cite{PoC}) afin de faire passer notre application modifiée pour celle d'origine et ainsi pouvoir l'exécuter sans problème.

Par la suite, cette faille de sécurité a été corrigée sur la plupart des appareils mobiles,
rendant impossible toute modification sur \textbf{Castle Defense}.
D'un autre coté, le PoC n'était pas nécessaire sur l'application \textbf{Checkers},
nous permettant donc de signer simplement l'application avec la clé de debug d'Android.


\subsection{Code Natif}

% TODO
Code natif dans de nombreuses applications.
Assembleur ARM.
Dur à comprendre et modifier.


\subsection{Un code obscur}

% TODO
Obscurtion du code de Checkers.
Suppression à la main des erreurs de décompilation.
Analyse.
Refactoring des variables et méthodes.

