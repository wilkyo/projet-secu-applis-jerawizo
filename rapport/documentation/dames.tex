
\subsection{Commencer avec des Dames}

Afin d'être à notre avantage, nous avons voulu faire commencer les noirs,
sous-entendu le joueur humain en début de partie, avec uniquement des dames.
Pour cela, il a fallu inverser les placements respectifs des pions et dames du joueur noir dans la méthode a.a() (correspondant à initPlateau).

\begin{figure}[!h]
\begin{verbatim}
    const/high16 v0, -0x10
	# Interversion des pions et dames des noirs v1 et v0
    iput v1, p0, Lcom/xxogli/xadroid/checkers/a;->f:I
    iput v0, p0, Lcom/xxogli/xadroid/checkers/a;->g:I
\end{verbatim}
    \caption{Interversion des pions et dames des noirs}
\end{figure}

La variable a.f correspond aux pions et a.g correspond aux dames.
v0, que l'on place sur les dames, contient le placement initial des noirs.
Pour les pions noirs, on place v1 = 0 qui correspond à aucun pion placé.
Ainsi, tous les pions noirs sont des dames.
