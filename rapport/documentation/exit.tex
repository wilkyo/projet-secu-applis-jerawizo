
\subsection{Ajout d'un bouton Exit}

Beaucoup d'applications n'ayant pas de bouton "Quitter" explicite et ne proposant parfois même pas de quitter (sans faire le bouton Home),
nous avons voulu en ajouter un dans le menu de l'application.

Pour cela, il a fallu modifier les méthodes onCreateOptionsMenu et onOptionsItemSelected de la classe \textit{Checkers}
afin de pouvoir ajouter le bouton et son action.

\begin{figure}[!h]
\begin{verbatim}
    const/4 v9, 0x7
...
    const-string v0, "Undo"
    invoke-interface {p1, v5, v4, v4, v0},
        Landroid/view/Menu;->add(IIILjava/lang/CharSequence;)
        Landroid/view/MenuItem;
    const-string v0, "Exit"
    invoke-interface {p1, v5, v9, v6, v0},
        Landroid/view/Menu;->add(IIILjava/lang/CharSequence;)
        Landroid/view/MenuItem;
    const-string v0, "Switch Side"
\end{verbatim}
    \caption{Ajout du bouton dans onCreateOptionsMenu}
\end{figure}

Lors de l'appui sur le bouton, un appel à la méthode finish() est exécuté.
Ainsi, les méthodes onPause (sauvegardant les données) et onStop (exécutant un System.exit()) sont appelées.

\begin{figure}[!h]
\begin{verbatim}
    const/16 v5, 0x7
...
    :cond_2
    if-ne v1, v3, :cond_42 # On va au test pour le bouton Exit
...
    :cond_42
    if-ne v1, v5, :cond_3 # On retourne au test suivant
    invoke-super {p0}, Landroid/app/Activity;->finish()V
    goto :goto_0
    :cond_3
\end{verbatim}
    \caption{Action du bouton dans onOptionsItemSelected}
\end{figure}
