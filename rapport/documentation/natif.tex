
\subsection{Des applications difficilement modifiables}

À la base, nous avions choisi de modifier l'application \textbf{Castle Defense}, un jeu de \textbf{Elite Games},
afin de nous accorder des avantages pendant la partie.
Nous avons donc décompilé l'application afin d'analyser où il faudrait appliquer les modifications,
mais nous ne trouvions rien de concluant parmis les quelques packages dont elle était composée.
Les seules pistes qu'on pouvait trouver concernaient la publicité et l'obtention de bonus en répondant à des enquêtes ou en suivant des liens divers.

Un fichier retenait tout de même notre attention: \texttt{libgame.so}.
Ce fichier est en fait une librairie codée en C/C++ puis compilée pour être utilisée par l'application en natif.
En analysant un peu la structure de cette librairie, elle se révéla contenir un grand nombre de méthodes intéressantes à modifier.
Nous nous sommes donc mis en quête des appels vers ces méthodes dans le code smali, mais sans succès.
Les seules méthodes appelées concernaient l'action de toucher l'écran et elles devaient probablement faire appel aux méthodes internes à la librairie.

La seule solution qui s'offrait à nous était donc d'attaquer cette librairie directement en héxadécimal.
Cette solution fut vite écartée et remplacée par la recherche d'une autre application ne contenant pas de code natif.
C'est ainsi que nous avons fini par trouver l'application \textbf{Checkers}, jeu de dames de \textbf{User inc.}.